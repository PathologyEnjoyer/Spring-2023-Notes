\documentclass[x11names,reqno,14pt]{extarticle}
\input{preamble}
\usepackage[document]{ragged2e}
\usepackage{amsmath}
\pagestyle{fancy}{
	\fancyhead[L]{Spring 2023}
	\fancyhead[C]{201C - Real Analysis}
	\fancyhead[R]{John White}
  
  \fancyfoot[R]{\footnotesize Page \thepage \ of \pageref{LastPage}}
	\fancyfoot[C]{}
	}
\fancypagestyle{firststyle}{
     \fancyhead[L]{}
     \fancyhead[R]{}
     \fancyhead[C]{}
     \renewcommand{\headrulewidth}{0pt}
	\fancyfoot[R]{\footnotesize Page \thepage \ of \pageref{LastPage}}
}

\newcommand{\seq}[2][\oo]{_{#2 = 1}^#1}
\newcommand{\barr}{{\bar{\R}}}
\newcommand{\cupk}[1][\oo]{\cup\seq[#1]{k}}
\newcommand{\cupi}[1][\oo]{\cup\seq[#1]{i}}
\newcommand{\cupn}[1][\oo]{\cup\seq[#1]{n}}
\newcommand{\bigcupk}[1][\oo]{\bigcup\seq[#1]{k}}
\newcommand{\bigcupi}[1][\oo]{\bigcup\seq[#1]{i}}
\newcommand{\bigcupn}[1][\oo]{\bigcup\seq[#1]{n}}
\newcommand{\capk}[1][\oo]{\cap\seq[#1]{k}}
\newcommand{\capi}[1][\oo]{\cap\seq[#1]{i}}
\newcommand{\capn}[1][\oo]{\cap\seq[#1]{n}}
\newcommand{\bigcapk}[1][\oo]{\bigcap\seq[#1]{k}}
\newcommand{\bigcapi}[1][\oo]{\bigcap\seq[#1]{i}}
\newcommand{\bigcapn}[1][\oo]{\bigcap\seq[#1]{n}}
\newcommand{\Dmn}{D_\mu\nu}
\newcommand{\Dnm}{D_nu\mu}
\newcommand{\loc}{loc}
\DeclareMathOperator{\Vol}{Vol}
\DeclareMathOperator{\diam}{diam}
\DeclareMathOperator{\BV}{BV}
\DeclareMathOperator{\Monp}{Mon^+}
\DeclareMathOperator{\AC}{AC}
\DeclareMathOperator{\Lip}{Lip}
\DeclareMathOperator{\Int}{int}
\title{201C - Functional Analysis}
\author{John White}
\date{Spring 2023}




\begin{document}

\section*{Lecture 1, 4/4/23}

We use the following two books:

\begin{enumerate}

\item Linear Analysis, by B. Bolob\'as 

\item Analysis, by E. Lieb and M. Loss

\end{enumerate}

We begin with metric spaces

\defn

Let $X$ be a nonempty set and let $\rho:X\times X\to[0,\oo)$. Then $\rho(x, y)$ is called \underline{a metric on $X$} if 

\begin{enumerate}[label=(\roman*)]

\item $\rho(x, y) \geq 0$ for all $x, y \in X$ and $\rho(x, y) = 0$ iff $x = y$. 

\item $\rho(x, y) = \rho(y, x)$ for all $x, y \in X$

\item $\rho(x, y) \leq \rho(x, z) + \rho(z, y)$ for any $x, y, z \in X$. This is called the \underline{triangle inequality}

\end{enumerate}

$\rho$ is also called a \underline{distance}. As in, $\rho(x, y)$ is the \underline{distance between $x$ and $y$}. 

\exm

Let $X = \R^n$, and define $\rho(x, y) = \left(\sum_{i=1}^n(x_i - y_i)^2\right)^{\frac{1}{2}}$. We can in fact replace $2$ in this expression with any real $r \geq 1$, or with $\oo$ (in which case we just take the maximum)

\exm

Let $X = C[a, b]$, the set of continuous $f:[a,b]\to\R$, and define $\rho(x, y) = \max_{x\in[a,b]}|f(x) - g(x)|$

\defn

Let $(X, \rho)$ be a metric space. For all $x \in X$ and $r > 0$, we defined the \underline{open ball centered at $x$ and having radius $r$} as
\[
B_r(x) \eqdef \{ y \in X \mid \rho(x, y) < r\}
\]
The closed ball is
\[
\bar{B_r}(x) \eqdef \{y \in X \mid \rho(x, y) \leq r\}
\]

\defn

Let $(X, \rho)$ be a metric space and let $A \subseteq X$. Then $a \in A$ is

\begin{enumerate}[label=(\roman*)]

\item an \underline{interior point} of $A$ if there is some $r>0$ such that $B_r(a) \subseteq A$

\item The set of all interior points of $A$ is called the \underline{interior of $A$} and is denoted by $\operatorname{int} A$, or $A^\circ$

\item A set $A$ is said to be $\underline{open}$ if $A = A^\circ$

\end{enumerate}

\exm

Let $X = \R^3$, $A = \{(x, y, 0) \mid x, y \in \R\}$. We can see that $A^\circ = \varnothing$.

\prop

For any $x, r$, $B_r(x)$ is open.

\proof

Let $y \in B_r(x)$. Let $r_1 = r - \rho(x, y) > 0$. 

Consider $z \in B_{r_1}(y)$. By the triangle inequality, $\rho(x, z) \leq \rho(z, y) + \rho(y, x) < r_1 + \rho(x, y) = r$. So $z \in B_r(x)$, so $B_{r_1}(y) \subseteq B_r(x)$, so $y$ is an interior point. $y$ was arbitrary, so we are done.

\qed

\defn

$A \subseteq X$ is \underline{closed} if $A^c = X\setminus A$ is open.

\defn

The point $x \in X$ is a \underline{limit point} of $A$ if there exists a sequence $\{x_n\}\seq{n} \subseteq A$ such that $\rho(x, x_n)\to0$ as $n\to\oo$

\defn

Let $\{x_n\}\subseteq X$, $x \in X$. Then we we say \underline{$x_n$ converges to $x$}, or $x_n\to x$, if $\rho(x_n, x)\to0$ as $n\to\oo$. In this case, $\{x_n\}\seq{n}$ is said to be \underline{convergent, with limit $x$}.

\thm

If a limit of a sequence $\{x_n\}\subseteq X$ exists, then it is unique. 

\proof

Think

\qed

\defn

A sequence $\{x_n\}\seq{n}\subseteq X$ is called a \underline{Cauchy sequence} if $\rho(x_n, x_m)\to0$ as $n,m\to\oo$. That is, for any $\varepsilon>0$, there exists an $N \in \N$, such that if $n, m \geq N$, then $\rho(x_n, x_m) < \varepsilon$

\thm

Any convergent sequence is Cauchy

\proof

Think

\qed

\defn

A metric space $(X,\rho)$ is called \underline{complete} if every Cauchy sequence converges to some point in $X$. A metric space which is not complete is called \underline{incomplete}.

\exm 

$X = \R^n$ or $X = C[a,b]$ with the metrics above are complete. 

\exm

$\Q$ is incomplete. 

\defn

Let $(X, \rho)$ and $(Y, \tilde{\rho})$ be metric spaces. Then $X$ and $Y$ are \underline{isometric} if there exist a bijection $f:X\to Y$ such that $\rho(x_1, x_2) = \tilde{\rho}(f(x_1),f(x_2))$ for all $x_1, x_2 \in X$. 

\defn

Let $(X, \rho)$ be a metric space and let $A, B \subseteq X$. Then we say that $A$ is \underline{dense in $B$} if $B \subseteq \bar{A}$, where $\bar{A} = \{$all limit points of $A\}$

\defn

Let $(X, \rho)$ and $(\tilde{X},\tilde{\rho})$ be metric spaces. Then $(\tilde{X},\tilde{\rho})$ is a \underline{completion of $(X, \rho)$} if
\begin{enumerate}[label=(\roman*)]

\item $X \subseteq \tilde{X}$, and $\tilde{\rho}(x, y) = \rho(x, y)$ for any $x, y \in X$

\item $X$ is dense in $\tilde{X}$ in the $\tilde{\rho}$ metric

\item $(\tilde{X},\tilde{\rho})$ is complete

\end{enumerate}

\thm 

Any incomplete metric space $(X,\rho)$ admits a completion which is unique up to isometry. 

\proof

Think

\qed

\thm (The nested ball theorem)

Let $(X,\rho)$ be a complete metric space, and let $\bar{B_n} = \bar{B_{r_n}}(x_n) \subseteq X$ be a sequence of nested closed balls (meaning $\bar{B_{n + 1}} \subseteq \bar{B_n}$) such that $r_n\to0$. Then $\bigcapn \bar{B_n} \neq \varnothing$. 

\proof

Consider the centers $\{x_n\}\seq{n} \subseteq X$. 

\claim $\{x_n\}$ is Cauchy

\proof

If $m \geq n$, then $\bar{B_m} \subseteq \bar{B_m}$, so $x_m \in \bar{B_m}$, so $\rho(x_m, x_n) \leq r_n$, so $\rho(x_n,x_m)\to0$ as $m,n\to\oo$.

\qed

\section*{Lecture 2, 4/6/23}

\defn

Let $(X, \rho)$ be a metric space, and let $A \subseteq X$. Then $A$ is \underline{nowhere dense} if $\operatorname{int}(\bar{A}) = \varnothing$

\defn

Let $(X, \rho)$ be a metric space, and $A$ a set. $A \subseteq X$ is \underline{of Baire first category} if $A = \bigcupn A_n$, where $A_n$ are nowhere dense. Otherwise, $A$ is \underline{of Baire second category}

\thm (Baire Category Theorem)

A complete space is of Baire second category. 

\proof

Towards contradiction, assume $X = \bigcupn A_n$, where $A_n \subseteq X$ are nowhere dense. 

Let $B_1 = B(X_1, 1)$ be a ball in $X$. 

Since $A_1$ is nowhere dense, there exists $\bar{B}_2 = \bar{B}(x_2, r_2) \subseteq \bar{B}_1$ such that $\bar{B}_2 \cap A_1 = \varnothing$. 

Without loss of generality, assume $r_2 < \frac{1}{2}$. Now there exists $\bar{B}_3 = \bar{B}(x_3, r_3) \subseteq \bar{B}_2$ such that $\bar{B}_2 \cap A_1 = \varnothing$. 

Without loss of generality, assume $r_3 < \frac13$.

At the $k$th step, there exists $\bar{B}_{k + 1} = \bar{B}(x_{k + 1}, r_{k + 1}) \subseteq \bar{B}_k$ such that $\bar{B}_{k + 1}\cap A_k = \varnothing$, $r_{k + 1} \leq \frac{1}{k + 1}$. 

By the nested balls theorem, $\bigcapn \bar{B}_n = \{x\}$. By construction, $x\not\in A_n$ for all $n\in\N$. So $X \neq \bigcupn A_n$, a contradiction. 

\qed

\defn

Let $(X, \rho)$ be a metric space and let $A \subseteq X$. $A$ collection $\{U_\alpha\}_{\alpha \in A}$ of open subsets of $X$ is an \underline{open cover of $A$} if $A \subseteq \bigcup_{\alpha\in A}U_\alpha$

A set $K \subseteq X$ is called \underline{compact} if any open cover of $K$ has a finite subcover. 

Equivalently, $K \subseteq X$ is compact if any sequence $\{x_n\} \subseteq K$ has a limit point $x \in K$.

\thm (Nested compact set theorem)

Let $(X, \rho)$ be a metric space and let $\{K_n\}\seq{n}$ be a sequence of nonempty and nested compact sets. Then $\bigcapn K_n \neq\varnothing$. 

\proof

Consider $\{x_n\}\seq{n}$ with $x_n \in K_n$. 

Note for all $n, x_n \in K_1$. Thus there exists a subsequence $\{x_{n_k}\}$ converging to some $x \in K_1$. 

We claim that $x \in \bigcapn K_n$.  

Fix $m \in \N$. $x_{n_m}, x_{n_{m + 1}}, x_{n_{m + 2}}, \dots \in K_m$. 

The only limit point is $x$, thus $x \in K_m$. 

\qed

\defn

Let $(X, \rho)$ be a metric space and let $A \subseteq X$. Then $A$ is bounded if $A \subseteq B(x, r)$ for some $x \in X, r > 0$. 

\thm

A compact set in $(X, \rho)$ is closed and bounded. 

\proof

think

\qed

\subsection*{\underline{Normed Spaces}}

\defn 

Let $X$ be a vector space. A function $\norm{\cdot}:X\to\R$ is called \underline{a norm on $X$} if
\begin{enumerate}[label=(\roman*)]

\item $\norm{x}\geq0$ for all $x \in X$.  Further, $\norm{x} = 0 \iff x = 0$
\item $\norm{\alpha x} = |\alpha|\norm{x}$ for all $x \in X, \alpha \in \C$.
\item $\norm{x + y} \leq \norm{x} + \norm{y}$

\end{enumerate}

($X, \norm{\cdot}$) is called \underline{a normed space}

Remark:

If one defines $\rho(x, y) = \norm{x - y}$, then $(X, \rho)$ is a metric space.

\section*{Lecture 3, 4/11/23}

\subsection*{\underline{Return to metric spaces}}

\defn Let $(X, d)$ be a metric space. Let $T:X\to X$ be a mapping. Then $T$ is said to be \underline{a contraction} if there exists $\alpha\in(0,1)$ such that $\rho(Tx,Ty) \leq \alpha\rho(x, y)$ for all $x, y \in X$. 

\thm (Contraction mapping principle)

Let $(X, \rho)$ be a complete metric space and let $T:X\to X$ be a contraction. Then $T$ has a unique fixed point, i.e. there exists a unique $x \in X$ such that $Tx = x$. 

\proof

Proven in 221A

\qed

\claim

$(0, 1) \neq \bigcupn\,[a_n, b_n]$, $[a_n, b_n]$ are disjoint. 

\proof 

Assume $(0, 1) = \bigcupn\, [a_n,b_n]$

We claim the set $X = \bigcupn \{a_n, b_n \} \cup \{0\} \cup \{1\}$ is closed. Thus $X$ is a complete metric space. Next, we claim that $\{a_n\},\{b_n\},\{0\},\{1\}$

Alternative proof:

Assume $(0, 1) = \bigcupn\,[a_n,b_n]$. 

We construct a function $f:(0,1)\to\R$ continuous that takes countably many values. blah blah blah

\qed

\subsection*{\underline{Return to normed spaces}}

\defn

A complete normed space is called a \underline{Banach Space}

\defn Let $X, Y$ be normed spaces on $K = \R$ or $\C$. A mapping $T:X\to Y$ is \underline{linear} if $T(\alpha x_1 + \beta x_2) = \alpha Tx_1 + \beta Tx_2$ for all $x_1, x_2 \in X$ and all $\alpha,\beta\in K$.

\defn

Let $X, Y$ be normed spaces on $K = \R$ or $\C$. Let $T:X\to Y$ be linear. Then
\begin{enumerate}

\item $T$ is \underline{bounded} if there exists $M>0$ such that for all $x \in X, \norm{Tx}\leq M\norm{x}$

\item The \underline{operator norm} is 
\[
\norm{T}\eqdef \sup_{x\in X, x\neq0}\frac{\norm{Tx}}{\norm{x}}
\]
If $T$ is bounded, then $\norm{T}\leq M$
\end{enumerate}

\defn

Let $X$ and $Y$ be normed spaces on $K = \R$ or $\C$. Let $T:X\to Y$ be linear. $T$ is \underline{continuous at $x_0 \in X$} if $x\to x_0$ implies $Tx\to Tx_0$.

\thm

Let $X, Y$ be normed spaces on $K = \R$ or $\C$. Let $T:X\to Y$ be linear. Then the following are equivalent:
\begin{enumerate}

\item $T$ is continuous at some $x_0 \in X$. 
\item $T$ is continuous at 0
\item $T$ is continuous on $X$
\item $T$ is bounded
\item $T$ is Lipschitz

\end{enumerate}

\proof\,

\subsection*{$(3)\Rightarrow (1),(2)$}

This is obvious

\subsection*{$(1)\Rightarrow(3)$}

Assume $T$ is continuous at some $x_0$. We want to show continuity at $y_0$. 

Suppose $(y_n)\to y$. Define $x_n = y_n - y_0 + x_0$. 

Note $x_n \to x_0$. So $Tx_n \to Tx_0$. Thus
\[
\norm{Ty_n - Ty_0} = \norm{Tx_n + Ty_0 - Tx_0 - Ty_0} \to 0
\]
Letting $x_0 = 0$, we get $(2)\Rightarrow(3)$

\subsection*{$(4)\Rightarrow(2)$}

Suppose $\norm{Tx} \leq M\norm{x}$ for all $x \in X$. 

Then as $x \to 0$, $\norm{Tx - T0} = \norm{Tx} \leq M\norm{x} \to M\norm{0} = 0$. 

\subsection*{$(2)\Rightarrow(4)$}

For all $\varepsilon>0$, there exists $\delta>0$ such that $\norm{x}<\delta\implies \norm{Tx}<\varepsilon$. 

Choose $\varepsilon = 1$. Then there exists $\delta>0$ such that $\norm{x} < \delta \Rightarrow \norm{Tx} < 1$

For any $x \in X, x\neq 0 \implies Tx = \frac{\norm{x}}{\delta}T(\frac{x}{\norm{x}}\delta)$

Set $\bar{x} = \frac{x}{\norm{x}} \delta$. 

Thus $\norm{Tx} = \frac{\norm{x}}{\delta}\norm{T\bar{x}} \leq \frac{\norm{x}}{\delta} = \frac{1}{\delta}\norm{x}$

So $\norm{Tx} \leq \frac{1}{\delta}\norm{x}$ for all $x \in X$, i.e. $M = \frac{1}{\delta}$.

\section*{Lecture 4, 4/13/23}

\subsection*{$(4)\Rightarrow(5)$}

$\norm{Tx_1 - Tx_2} = \norm{T(x_1 - x_2)} \leq M\norm{x_1 - x_2}$. Thus $T$ is Lipshitz. 

Clearly, $(5)\Rightarrow(3)$.

\qed

\defn

For $X, Y$ normed spaces, the set of all bounded linear operators $T:X\to Y$ is denoted by $B(X, Y)$. 

\thm Let $X, Y$ be normed spaces and let $T \in B(X, Y)$. Then 
\[
\norm{T} = \sup_{\norm{x}\leq1}\norm{Tx} = \sup_{\norm{x} = 1}\norm{Tx} = \sup_{\norm{x}<1}\norm{Tx}
\]

\proof

$\norm{T} = \sup_{x\in X, x\neq0}\frac{\norm{Tx}}{\norm{x}} = \sup_{x\in X, x \neq 0}\norm{T(\frac{x}{\norm{x}})}$. 

So, letting $\bar{x} = \frac{x}{\norm{x}}$, $\norm{T} = \sup_{x\in X, x\neq0}\norm{T\bar{x}} \leq \sup_{\norm{x}=1}\norm{Tx}$

Similarly, $\norm{T} \leq \sup_{\norm{x}\leq1}\norm{Tx}$

$\norm{T} = \sup_{x\in X, x\neq0}\frac{\norm{Tx}}{\norm{x}} \geq \sup_{x\in X, \norm{x} = 1}\frac{\norm{Tx}}{\norm{x}}$, so we get equality. 

We could prove $<$ using limits. 

\thm 

Let $X$ and $Y$ be normed spaces and let $T:X\to Y$ be linear. Then $T$ is bounded iff $T$ maps bounded sets to bounded sets.

\proof

Later

\qed

Note: $\norm{Tx} \leq \norm{T}\norm{x}$. Also, a convergent sequence in a metric space is bounded. 

\thm Let $X$ and $Y$ be normed spaces. Then $B(X, Y)$ is a normed space endowed with the operator norm $\norm{T}$. Moreover, if $Y$ is Banach, then $B(X, Y)$ is Banach.

\proof

$B(X, Y)$ is a vector space:

Choose $\alpha,\beta \in K, T_1, T_2 \in B(X, Y)$. Then $\alpha T_1 + \beta T_2$ is linear and continuous. Hence, $\alpha T_1 + \beta T_2 \in B(X, Y)$. 

Now we show that $(B(X, Y), \norm{T})$ is a normed space. 

\begin{enumerate}

\item $\norm{T} \geq 0$ for all $T \in B(X, Y)$ clearly, and $\norm{T} = 0$ means $Tx = 0$ for any $x \in X$, so $T = 0$. 

\item $\norm{\alpha T} = |\alpha|\norm{T}$

\item Triangle inequality: Choose $x \in B(X, Y)$. Then
\begin{align*}
\norm{(T_1 + T_2)x} & = \norm{T_1x + T_2x} \\
						  & \leq \norm{T_1x} + \norm{T_2x} \\
						  & \leq \norm{T_1}\norm{x} + \norm{T_2}\norm{x}\\
						  & = \norm{x}(\norm{T_1} + \norm{T_2}) \\
\end{align*}
Thus, $\norm{T_1 + T_2} = \sup_{x\in X, x\neq 0}\norm{(T_1 + T_2)x} \leq \sup_{\norm{x}\leq 1}(\norm{T_1} + \norm{T_2})\norm{x} \leq \norm{T_1} + \norm{T_2}$

\end{enumerate}

Now assume $Y$ is a Banach space. 

Let $\{T_n\} \subseteq B(X, Y)$ be a Cauchy sequence.

We construct an operator $T:X\to Y$ as follows. 

For all $x\in X$, $\{T_nx\}$ is a Cauchy sequence in $Y$. 

Since $Y$ is complete, $\{T_nx\}$ converges to some $Tx$

We want to show $T$ is linear and bounded. 

For all $n \in \N$, $\alpha,\beta\in K$, $x_1, x_2 \in X, T_n(\alpha x_1 + \beta x_2) = \alpha Tx_1 + \beta Tx_2$. 

Thus $T(\alpha x_1 + \beta x_2) = \alpha Tx_1 + \beta Tx_2$

Now we show $T$ is bounded.

Since a Cauchy sequence in a metric space is bounded, $\norm{T_n} \leq M$ for all $n \in \N$. 

For all $x \in \bar{B(0, 1)}, (T_nx)\to Tx$. 

Thus $\norm{Tx} = \lim_{n\to\oo}\norm{T_nx} \leq \lim_{n\to\oo}M\norm{x} \leq M$. 

Hence $\norm{T}\leq M$. 

\,

Finally, we show $(T_n)\to T$. 

Choose $\varepsilon>0$. There exists $N$ such that for all $n \geq N$, for all $k \in \N$, $\norm{T_n - T_{n + k}}<\varepsilon$. 

For all $x \in \bar{B(0, 1)}$, $\norm{T_nx - T_{n + k}x} \leq \varepsilon\norm{x}$. 

Fix $n \geq N$, and let $k\to\oo$ to get $\norm{T_nx - Tx} \leq \varepsilon\norm{x}$. Thus, $\norm{T_n - T} \leq \varepsilon$ for all $n \in \N$> 

So $(T_n) \to T$ in $B(X, Y)$

\qed

\defn

Let $X$ be a normed space over a field $K = \R$ or $\C$. Then $B(X, K) = X^\star$ is called \underline{the dual space of $X$}. $T \in X^\star$ is called a \underline{functional}.

Special case: $X = L^p(\Omega)$, we will characterize $X^\star$. 

When $1 \leq p < \oo$, then $(L^p(\Omega))^\star = L^q(\Omega)$, where $\frac{1}{p} + \frac{1}{q} = 1$ (when $p = 1, q = \oo)$

\thm

Let $X$ be a normed space and $Y$ a Banach space. Assume $A \subseteq X$ is a dense subspace of $X$ and $T\in B(A, Y)$. Then $T$ admits a unique extension $\bar{T}\in B(X, Y).$ Moreover, $\norm{\bar{T}} = \norm{T}$. 

\proof

Chose $x \in X\setminus A$. 

There exists $\{a_n\} \in A$ converging to $x$ because $A$ is dense. Define $Tx = \lim_{n\to\oo}Ta_n$. 

\,

First we show this limit exists. 

Note $\{Ta_n\}$ is Cauchy, since $\norm{Ta_n - Ta_m} \leq \norm{T}\norm{a_n - a_m} \to 0$. 

Since $Y$ is Banach, $\{Ta_n\}$ conveges. 

Now we show the limit does not depend on choice of sequence $\{a_n\}$. 

Assume $\{b_n\}\to X$. Then
\[
\norm{Ta_n - Tb_n} = \norm{T} \norm{a_n - b_n} \leq \norm{T}(\norm{a_n - x} + \norm{b_n - x}) \to 0
\]
Thus $\lim_{n\to\oo}Ta_n = \lim_{n\to\oo}Tb_n$

Next we show $\bar{T}$ is linear.

\,

Choose $\alpha,\beta \in K, x_1, x_2 \in X$. 

Let $\{a_n\}\to x_1, \{b_n\}\to x_2$ where $\{a_n\},\{b_n\}\subseteq A$. 

$\alpha a_n \to \alpha x_1, \beta b_n \to \beta x_2$

Thus $\alpha a_n + \beta b_n \to \alpha x_1 + \beta x_2$

So
\[
\bar{T}(\alpha x_1 + \beta x_2) = \lim_{n\to\oo}T(\alpha x_1 + \beta x_2) = lim_{n\to\oo}T(\alpha x_n) + \lim_{n\to\oo}T(\beta b_n) = \alpha \bar{T}x_1 + \beta \bar{T}x_2
\]

\qed

\section*{Lecture 5, 4/18/23}

\thm Let $X$ be a normed space, and $Y$ a Banach space. Assume $A \subseteq X$ is a dense subspace of $X$ and $T \in B(A, Y)$. Then $T$ admits a unit extension $\bar{T}\in B(X, Y)$. Moreover, $\norm{\bar{T}} = \norm{T}$

\proof

If $x\not\in A$, choose $a_n \in A$ such that $a_n\to x$. Define $\bar{T}x =\lim_{n\to\oo}Ta_n$. Then $\{Ta_n\}$ is Cauchby, and $Y$ is Banach. 

\begin{enumerate}

\item $\bar{T}$ is linear

\item $\bar{T}$ is bounded, with $\norm{\bar{T}} \leq \norm{T}$. 

For any $x \in B(0, 1)$, there exists $a_n\in B(0, 1)$ such that $a_n\to x, a_n \in A$. 

\begin{align*}
\norm{\bar{T}x}& \leq \norm{\bar{T}x - \bar{T}a_n} + \norm{Ta_n} \\ 
& \leq \norm{\bar{T}x - Ta_n} + \norm{T}\norm{a_n} \\
& \leq \norm{\bar{T}x - Ta_n} + \norm{T} 
\end{align*}
As $n\to\oo$, we get $\norm{\bar{T}x} \leq \norm{T}$ for all $x \in B(0, 1)$. Thus $\norm{T} = \sup_{x\in B(0, 1)}\norm{\bar{T}x} \leq \norm{T}$


\end{enumerate}

\qed

\thm

Let $X$ be a normed space, and let $A = \{x_1, \dots, x_n\}$, where $\{x_i\}_{i=1}^n$ is linearly independent. Then
\begin{enumerate}

\item $A$ is closed

\item Let $a_k = \sum_{i=1}^n\alpha_ix_i \in A$, and $a_k \to x \in X$. Then by 1, $x \in \langle A \rangle$, i.e. $x = \sum_{i=1}^n\alpha_ix_i$. Then $\alpha_i^k\to\alpha_i$ for $i = 1, 2, \dots, n$. 

(Convergence in A is equivalent to convergence of coordinates).

\end{enumerate}

\proof

First, assume $X$ is a Euclidean space. Define $T:\R^n\to\langle A \rangle$ given by $T(c_1,\dots, c_n) = c_1x_1 + c_2x_2 + \cdots + c_nx_n$. 

Note that
\begin{itemize}

\item $T$ is linear on $\R^n$. 

\item $T$ is continuous:
\begin{align*}
\norm{T(c_1, \dots, c_n)} & = \norm{c_1x_1 + \cdots + c_nx_n} \\
& \leq |c_1|\norm{x_1} + \cdots + |c_n|\norm{x_n} \\
& \leq |c|\max_i\norm{x_i}
\end{align*}

\begin{align*}
\norm{T(c_1, \dots, c_n) - T(b_1, \dots, b_n)} & = \norm{T(c_1 - b_1, \dots, c_n - b_n)} \\
& \leq \sum_{i=1}^n|c_i - b_i|\norm{x_i} \\
& \leq n|c - b|(\max_i\norm{x_i}) \\
& \leq |c - b|\left(\sum_{i=1}^n\norm{x_i}^2\right)^{\frac12}
\end{align*}

Thus $T$ is continuous.

\end{itemize}

$S^{n - 1} = \del B(0, 1) = \{(c_1, \dots, c_n) \mid \norm{c} =1 \}$. 

$S^{n - 1}$ is compact, so $f(c) = \norm{Tc}, c \in S^{n - 1}$ attains its minimal value on $S^{n - 1}$. 

$|\norm{x} - \norm{y}| \leq \norm{x - y}$ for all $x,  y \in X$. 

$\min_{c\in S^{n - 1}}\norm{Tc} = \norm{Tc_0}$, where $c_0 = (c_1^\circ, \dots, c_n^\circ) \in S^{n - 1}$

I claim now that $\norm{Tc_0} > 0$. 

$Tc_0 = c_0^1x_1 + \dots + c_0^nx_n \neq 0$, so $\norm{Tc_0} > 0$. 

$r = \norm{Tc_0} = \min_{c\in S^{n - 1}}\norm{Tc}>0$

$\norm{Tc} \geq r > 0$ for all $c \in S^{n - 1}$. 

For all $b \in\R^n, b\neq0$, 
\begin{align*}
\norm{Tb} & = \norm{|b|T(\frac{b}{|b|})} \\
& = |b|\norm{T(\frac{b}{|b|})} \\
& \geq r|b|
\end{align*}

Thus $\norm{Tb} \geq rb$ for all $b \in \R^n$. 

Assume now $a_k = \sum_{i=1}^n\alpha_i^kx_i\to x \in X$. 

$a_k = T(\alpha_1^k,\dots,\alpha_n^k), x = \sum_{i=1}^n\alpha_ix_i$.

Since $a_k\to a$, $\norm{a_k} \leq M$. 

So $|(\alpha_1^k,\dots,\alpha_n^k)| \leq \frac{1}{r}\norm{T(\alpha_1^k,\dots,\alpha_n^k)} \leq \frac{M}{r}$

By Bolzano-Weierstrass, there exists $(\alpha_1^{k_\ell}, \dots, \alpha_n^{k_\ell}) \to (\alpha_1,\dots,\alpha_n)$ as $\ell\to\oo$

Since $T$ is continuous, 
\[
\alpha_{k_\ell} = T(\alpha_1^{k_\ell},\dots,\alpha_n^{k_\ell}) = \alpha_1x_1 + \cdots + \alpha_nx_n \in \langle A \rangle
\]
\qed

\thm[Uniform Boundedness Principle, Banach-Steinhaus Theorem]

Let $X$ be a banach space, and let $Y$ be a normed space. Let $\{T_\alpha\}_{\alpha\in A} \subseteq B(X, Y)$ be a family such that the set $\{T_\alpha x\}_{\alpha\in \bigtriangleup}$ is bounded for all $x \in X$. Then $\{T_\alpha\}$ is bounded in $B(X, Y)$, i.e. there exists $M > 0$ such that $\norm{T_\alpha} \leq M$ for all $\alpha\in \bigtriangleup$

\proof

Consider the sets $A_n = \{x\in X \mid \norm{T_\alpha x} \leq n \forall \alpha\in \bigtriangleup\}\subseteq X$

\begin{enumerate}

\item $\bigcupn A_n = X$. 

\item Each $A_n$ is closed in $X$. 

Let $x_k \in A_n, x_k \to x \in X$. 

We want to show $x \in A_n$. 

$\forall\alpha\in\bigtriangleup, \norm{T_\alpha x_k}\leq n, k = 1, 2, \dots$

As $k\to\oo$, we have that $\norm{T_\alpha x}\leq n$

So $x \in A_n$. 

By Baire Category Theorem, there exists some $A_N$ not nowhere dense. 

$\Int \bar{A_n} = \Int A_n \neq \varnothing$.

So there exists $x_0 \in X, r > 0$  such that $B(x_0, r) \subseteq A_N$. 

Thus $\bar{B}(x_0, \frac{r}{2}) \subseteq A_N$. Let $R = \frac{r}{2}$

$\norm{T_\alpha(\bar{B}(x_0, R)}\leq N\forall\alpha\in\bigtriangleup$

Let $x \in X, \norm{x}<R$. $x = x_0 + x - x_0$

$x_0 + x \in\bar{B}(x_0, R)$. 

\begin{align*}
\norm{T_\alpha x} & = \norm{T_\alpha(x + x_0) - T_\alpha x_0} \\
& \leq \norm{T_\alpha(x + x_0)} + \norm{T_\alpha x_0} \\
& \leq 2N
\end{align*}
$\forall x \in\bar{B}(0,R), \forall\alpha\in\bigtriangleup,\norm{T_\alpha x}\leq 2N$

$\forall x\in\bar{B}(0,1),\forall\alpha\in\bigtriangleup,\norm{T_\alpha x}= \norm{\frac{1}{R}T_\alpha (Rx)} = \frac{1}{R}\norm{T_\alpha(Rx)}\leq\frac{1}{R}2N = \frac{2N}{R} = M$

$\forall x\in\bar{B}(0,1),\forall\alpha\in\bigtriangleup,\norm{T_\alpha x}\leq M \implies \norm{T_\alpha}\leq M$






\end{enumerate}












\end{document}