\documentclass[x11names,reqno,14pt]{extarticle}
\input{preamble}
\usepackage[document]{ragged2e}
\usepackage{epsfig}

\pagestyle{fancy}{
	\fancyhead[L]{Spring 2023}
	\fancyhead[C]{236B}
	\fancyhead[R]{John White}
  
  \fancyfoot[R]{\footnotesize Page \thepage \ of \pageref{LastPage}}
	\fancyfoot[C]{}
	}
\fancypagestyle{firststyle}{
     \fancyhead[L]{}
     \fancyhead[R]{}
     \fancyhead[C]{}
     \renewcommand{\headrulewidth}{0pt}
	\fancyfoot[R]{\footnotesize Page \thepage \ of \pageref{LastPage}}
}
\newcommand{\pmat}[4]{\begin{pmatrix} #1 & #2 \\ #3 & #4 \end{pmatrix}}
\newcommand{\A}{\mathbb{A}}
\newcommand{\B}{\mathbb{B}}
\newcommand{\fin}{``\in"}
\DeclareMathOperator{\Perm}{Perm}
\DeclareMathOperator{\pdim}{pdim}
\DeclareMathOperator{\gldim}{gldim}
\DeclareMathOperator{\lgldim}{lgldim}
\DeclareMathOperator{\rgldim}{rgldim}
\DeclareMathOperator{\idim}{idim}
\DeclareMathOperator{\supp}{supp}
\newcommand{\Rmod}{R-\text{mod}}
\newcommand{\RMod}{R-\text{Mod}}
\newcommand{\onto}{\twoheadrightarrow}
\newcommand{\barf}{\bar{f}}
\newcommand{\D}{\mathbb{D}}
\newcommand{\into}{\hookrightarrow}
\renewcommand{\P}{\mathbb{P}}
\renewcommand{\E}{\mathbb{E}}
\DeclareMathOperator{\Ext}{Ext}
\DeclareMathOperator{\SAb}{SAb}
\DeclareMathOperator{\PAb}{PAb}
\DeclareMathOperator{\Ob}{Ob}
\DeclareMathOperator{\Kom}{Kom}
\DeclareMathOperator{\Mor}{Mor}

\newcommand{\exactlon}[5]{
		\begin{tikzcd}
			0\ar[r]&#1\ar[r,"#2"]& #3 \ar[r,"#4"]& #5 \ar[r]&0
		\end{tikzcd}
}

\title{236B - Homological Algebra}
\author{John White}
\date{Spring 2023}


\begin{document}

\section*{Lecture 1, 4/3/23}

We will begin thinking about homological algebra in a more categorical way. We use ``methods of homological algebra,'' by Gelfond-Manin (Gelfond $\&$ Manin?). Here is a brief overview of some things we will see during the course:

\begin{itemize}

\item Additive and Abelian categories

\item Example: category of sheaves of Abelian groups on a topological space

\item Derived category (objects: complexes of objects in a given Abelian category, morphisms: a quasi-isomorphism of complexes becomes an isomorphism of objects)

\item Derived functors, $f_*, f^*, f_{!}, f^{!}, \D, \cdots$, $\Hom$ for sheaves

\item Triangulated categories

\item $t$-structures

\item Examples: perverse sheaves


\end{itemize} 


Let's remind ourselves of some definitions:

\subsection*{Additive $\&$ Abelian categories}

\exm

\begin{enumerate}[label=(\alph*)]

\item $\Ab$, the category of Abelian groups with group homomorphisms.

\item $\RMod$, the category of $R$-modules, with $R$-module homomorphisms as morphisms.

\item $\SAb, \PAb$, the categories of sheaves of Abelian groups and presheaves of Abelian groups

\item Sheaves of modules over a ringed space 

\item (quasi)-coherent sheaves (ask Zhao)

\end{enumerate}

\defn

An \underline{Abelian category} contains the following information:

\begin{enumerate}

\item Any hom-set $\Hom_{\ms{C}}(X, Y)$ is an Abelian group (+), and the composition of morphisms is bi-additive

In particular:

\begin{itemize}

\item $\Hom_{\ms{C}}$ is a functor $\ms{C}^\circ\times\ms{C}\to\Ab$. We notate the first factor with the $^\circ$

\item $0\in\Hom_{\ms{C}}(X, Y)$ for any objects $X, Y$ of $\ms{C}$

\end{itemize}

\item There exists a zero object $0\in\ms{C}$, that is an object such that $\Hom_{\ms{C}}(0,0) = 0$. 

This gives: $\Hom_{\ms{C}}(0, X) = 0$, $\Hom_{\ms{C}}(X, 0) = 0$ for all objects $X$ of $\ms{C}$.

We know that $\Hom_{\ms{C}}(0, 0)$ consists of one object. In particular, it must be $\Id_0 = 0$. So
\[
(\begin{tikzcd} 0\ar[r, "f"] & X \end{tikzcd} ) = (\begin{tikzcd} 0 \ar[r, "\Id_0"] & 0 \ar[r, "f"] & X \end{tikzcd}) = (\begin{tikzcd} 0\ar[r, "0"] & X \end{tikzcd})
\]
Any two zero objects are isomorphic. 

\item For any $X_1, X_2 \in \ms{C}$, there exists an object $Y$ and morphisms 
\[
\begin{tikzcd} X_1 \ar[r, bend left = 30, "i_1"] &\ar[l, bend left = 20, "p_1"] Y \ar[r, bend right = 30, "p_2"'] & \ar[l, bend right = 20, "i_2"'] X_2 \end{tikzcd}
\]
such that 
\begin{align*}
p_1i_1 & = \Id_{X_1} \\
p_2i_2 & = \Id_{X_2} \\
i_1p_1 + i_2p_2 & = \Id_Y \\
p_2i_1 = p_1i_2 & = 0 \\
\end{align*}

\lem
 We have cartesian diagram 
\[
\begin{tikzcd}
Y'\ar[dr, dotted, "\exists!\varphi"]\ar[drr, "p_1'"]\ar[ddr, "p_2'"'] &  &  \\
& Y \ar[r, "p_1"] \ar[d, "p_2"] & X_1 \ar[d] \\
& X_2 \ar[r] & 0  \\
\end{tikzcd}
\]
That is, for any $Y'$, with morphisms $p_1', p_2'$ as in the diagram, there is a morphism from $Y'$ to $Y$ making the diagram commute. Similarly, we have co-cartesian diagram
\[
\begin{tikzcd}
Y & \ar[l, "i_1"'] X_1 \\
X_2 \ar[u, "i_2"] & 0 \ar[l] \ar[u] \\
\end{tikzcd}
\]

\proof 

We need to construct $\varphi:Y'\to Y$ such that $p_1' = p_1\varphi$ and $p_2'=p_2\varphi$

Take $\varphi = i_1p_1' + i_2p_2'$. Then 
\[
p_1\circ\varphi = \underbrace{p_1i_1}_{=\Id_{X_1}}p_1' + \underbrace{p_1i_2}_{=0}p_2' = p_1'
\]
The uniqueness of $\varphi$ can be verified as an exercise

\qed

\defn

An \underline{additive category} is one which satisfies only the first three of these axioms

\item To state the 4th axiom of Abelian categories, we need more notations:

\defn Let $A_1,A_2$ be objects of $\ms{C}$, and let $\varphi:X\to Y$.

\begin{enumerate}
\item A \underline{kernel} of $\varphi$ is a morphism $i:Z\to X$ such that 
\begin{enumerate}[label=(\alph*)]

\item $\varphi\circ i = 0$

\item For all $i':Z'\to X$ such that $\varphi\circ i' = 0$, there is a unique $g:Z'\to Z$ such that $i'=i\circ g$. 
\[
\begin{tikzcd}
 & Z' \ar[d, "i'"] \ar[dr, "0"] \ar[dl, dotted, "g"'] & \\
Z \ar[r, "i"] & X \ar[r, "\varphi"] & Y \\
\end{tikzcd}
\]

\end{enumerate}

\item A \underline{cokernel} is a kernel but with the arrows reversed

\end{enumerate}

Exercise: Verify that 1 is equivalent to the following: for all $Z'\in\ms{C}$, 
\[
\begin{tikzcd}
0 \ar[r]& \Hom(Z',Z) \ar[r, "i_*"] & \Hom(Z', X) \ar[r, "\varphi_*"] & \Hom(Z', Y) 
\end{tikzcd}
\]
is exact, and similarly for cokernel

With all that, we are ready for: 

\item For any $\varphi:X\to Y$, there exists a sequence of morphisms 
\[
\begin{tikzcd}
K\ar[r, "k"] & X \ar[r, "i"] & I \ar[r, "j"] & Y \ar[r, "c"] & K'
\end{tikzcd}
\]
such that
\begin{enumerate}[label=(\alph*)]

\item $j\circ i = \varphi$

\item $k = \ker \varphi, k' = \coker\varphi$

\item $I = \coker k = \ker c$

\end{enumerate}

\end{enumerate}

This finishes the definition

\section*{Lecture 2, 4/5/23}

\subsection*{Sheaves}

Here are some examples of sheaves from complex analysis

\exm\,

\begin{enumerate}[label=(\alph*)]

\item The set of holomorphic functions on $\mbb{P} = \C \sup \{\oo\}$

For each open subset $U$ of $\mbb{P}$, we can consider the ring of holomorphic function $f:U\to\C$, $\ms{H}(U)$. 

The collection of $\{(f, V) \mid V \text{ open}, f \in \ms{H}(V) \}$ is called the sheaf $\ms{O}$ of holomorphic functions on $\mbb{P}$.

\item The sheaf of solutions of a linear ODE

Let $U \subseteq \mbb{P}$ be open, and let $a_i(z) \in \Gamma(U,\ms{O})$ (in this context this will wind up meaning $\ms{H}(U)$), $i = 0,1,\dots,n - 1$. 

Denote by $S$ the collection of $(V, f)$ such that $V \subseteq U$ is open, and $f$ is holomorphic in $V$ such that 
\[
Lf \eqdef \frac{d^nf}{dz^n} + \sum_{i=0}^{n - 1}a_i(z)\frac{d^if}{dz^i} = 0
\]

Let $\Gamma(V) = \{f \in \ms{H}(V) \mid Lf = 0 \}$. When $V$ is connected and simply connected, it is a basic result of ODEs that $\Gamma(V)\cong \C^n$

In general, it may have to do with the topology of $V$. For example, if $U = \C\setminus\{0\}$, $L = \frac{d^2}{dz^2} + \frac{1}{z}\frac{d}{dz}$, the solutions are $c_1\log(z) + c_2$ for any ``branch" of $\log(z)$, but $\Gamma(V) = \{\text{constant}\}$. This is related to the Riemann-Hilbert correspondence (whatever that is)

\end{enumerate}

\defn\,

\begin{enumerate}[label=(\alph*)]

\item A \underline{presheaf of sets $\mc{F}$ on a topological space $Y$} consists of the following data:

\begin{itemize}

\item A set $\mc{F}(U)$ for any open $U \subseteq Y$

\item For any open $V \subseteq U$, a (restriction) map $\gamma_{U,V}:\mc{F}(U)\to\mc{F}(V)$ such that $\gamma_{V,V} = \Id_{\mc{F}(V)}$, and if $W \subseteq V \subseteq U$, then $\gamma_{V,W}\circ\gamma_{U,V} = \gamma_{U,W}$. When there is ambiguity about which sheaf $\gamma$ belongs to, we further specify with $\gamma^{\mc{F}}$

\end{itemize}

\item A presheaf $\mc{F}$ is a \underline{sheaf} if: 

\begin{itemize}

\item For any open covering $U = \cup_{i\in I}U_i$ and $s_i \in \mc{F}(U_i)$ such that for all $i, j \in I$, 
\[
\gamma_{U_i, U_i \cap U_j}(s_i) = \gamma_{U_j, U_i \cap U_j}(s_j)
\]
then there exists a unique $s\in \mc{F}(U)$ such that $s_i = \gamma_{U,U_i}(s)$ for all $i$. 

\end{itemize}

\item A \underline{morphism of presheaves} $f:\mc{F}\to\mc{G}$ is a family of maps $f(U):\mc{F}(U)\to\mc{G}(U)$ for all open $U \subseteq Y$, such that for all open $V \subseteq U$, the diagram
\[
\begin{tikzcd}
\mc{F}(U) \ar[d, "\gamma^{\mc{F}}_{U,V}"'] \ar[r, "f(U)"] & \mc{G}(U) \ar[d, "\gamma^{\mc{G}}_{U, V}"] \\
\mc{F}(V) \ar[r, "f(V)"] & \mc{G}(V) \\
\end{tikzcd}
\]
commutes.

A \underline{morphism of sheaves} is a morphism of the underlying presheaves



\end{enumerate}

A presheaf $\mc{F}$ of groups/rings/$\mbb{F}$-vector spaces is a presheaf such that $\mc{F}(U)$ is a group/ring/$\mbb{F}$-vector space. Then $\gamma_{U,V}$ is a morphism of groups/rings/$\mbb{F}$-vector spaces. 

\,

Let $\mc{F},\mc{G}$ be two Abelian presheaves (meaning presheaves of Abelian groups), and let $f:\mc{F}\to\mc{G}$ be a morphism of Abelian presheaves. 

Let $K(U) = \ker(f(U))$, $C(U) = \coker(f(U))$ with natural restrictions.

\defn

A sequence of presheaves
\[
\begin{tikzcd}
\mc{F}\ar[r, "f"] & \mc{G} \ar[r, "g"] & \mc{H}
\end{tikzcd}
\]
is \underline{exact} if, for all open $U$, 
\[
\begin{tikzcd}
\mc{F}(U) \ar[r, "f(U)"] & \mc{G}(U) \ar[r, "g(U)"] & \mc{H}(U)
\end{tikzcd}
\]
is exact. 

So, presheaves of Abelian groups, with morphisms of Abelian presheaves, is an Abelian category.

How about Abelian sheaves? 

Let $f:\mc{F}\to\mc{G}$ be a morphism of Abelian sheaves. Once again, let $K(U) = \ker(f(U)), C(U) = \coker(f(U)) = \frac{\mc{G}(U)}{f(U)(\mc{F}(U))}$. 

\prop\,

\begin{enumerate}[label=(\alph*)]

\item The kernel $K$ is an Abelian sheaf

\item The cokernel $C$ is always a presheaf, but might not be a sheaf.


\end{enumerate}

\section*{Lecture 3, 4/7/23}

\proof\,

\begin{enumerate}[label=(\alph*)]

\item Let $U = \bigcup U_i$, $s_i \in K(U_i)$ agree on pairwise intersections. As $K(U_i) \into \mc{F}(U_i)$, there exists a unique $s \in \mc{F}(U)$ such that $s|_{U_i} = s_i \in \mc{F}(U_i)$ for all $i$.

Note that $f(s)|_{U_i} = f(s|_{U_i}) = 0$, so $f(s) = 0 \in \mc{G}(U)$. Here, we are using the uniqueness of gluing in $\mc{G}$.  

Then $s \fin K(U)$

\item First, we give an example of a cokernel which is a presheaf, but not a sheaf. Let $Y = \C\setminus\{0\}$, $f:\mc{O}_Y\to\mc{O}_Y$ given by $\varphi\mapsto\frac{d\varphi}{d\zeta}$. 

\begin{itemize}

\item For any $y \in Y$, there exists a neighborhood $V_y \ni y$ such that $\coker f(V_y) = 0$. That is, for every $f\in \mc{H}(V_y)$, there is a $g \in \mc{H}(V_y)$ so that $\frac{dg}{d\zeta} = f$ in $V_y$ (every point admits a simply connected neighborhood)

\item However, $\coker f(Y) \cong \C$: $\Psi = \sum_{i=-\oo}^\oo a_i\zeta^i = \frac{d\varphi}{d\zeta}$ has a solution iff $a_{-1} = 0$. This is because $\frac{1}{z}$ is defined on $Y$. (keyword: vanishing cycle and v-filtration). 

Hence this violates the uniqueness: any $\bar{\Psi} \in \coker f(Y)$ restricts to $0 \in \coker f(V_y)$. However, the $V_y$ cover $Y$. So, we have an open cover, with sections of each, which agree on intersections, but which do not have a \underline{unique} global section to which they all glue. So, this is not a sheaf.

\end{itemize}

\end{enumerate}

\qed

\subsection*{Sheafification}

Denote by $\SAb$ the additive category of abelian sheaves on a fixed topological space $M$. We have the inclusion functor $\iota:\SAb\to\PAb$. 

\prop

$\iota$ admits a left adjoint $s:\PAb\to\SAb$, i.e. 
\[
\Hom_{\SAb}(sX, Y) \cong \Hom_{\PAb}(X, \iota Y)
\]
and this isomorphism is natural in both $X$ and $Y$.

\proof

Let 
\[
s(X)(U) \eqdef \frac{\{\{(U_i, e_i)\}_{i\in I} \mid U = \bigcup_{i\in I} U_i, e_i \in X(U_i), e_i|_{U_i \cap U_j} = e_j|_{U_i \cap U_j}\}}{\sim}
\]
Where $\{(U_i, e_i)\} \sim \{(U_j', e_j)\}$ if there exists $\{U_k''\}$ refining $\{U_i\}, \{U_j'\}$ and $e_i|_{U_k''} = e_j'|_{U_k''}$ for $U_k'' \subset U_i \cap U_j'$. 

Define $\gamma_{U,V}:sX(U)\to sX(V)$, for $V \subseteq U$, by 
\[
\gamma_{U,V}[\{(U_i,e_i)\}] = [\{(U_i \cap V, e_i|_{U_i \cap V})\}]
\]

There is a lot to verify; see [GM] 2.5.13

\qed

\exm

For $\coker f$ from previous example, now $[\bar{\Psi}] = [0]$, since, when we restrict to $V_y$, $\bar{\Psi}$ becomes 0. So $\coker f = 0$

\,

With this modification, $\SAb$ is an abelian category!

\prop

Let $\varphi:X\to Y$ be a morphism of abelian sheaves, and let 
\[
\begin{tikzcd}
K\ar[r, "k"] & \iota X \ar[r, "i"] & I \ar[r, "j"] & \iota Y \ar[r, "c"] & K'
\end{tikzcd}
\]
be the canonical decomposition of $\iota(\varphi)$ in (the abelian category!) $\PAb$. Then
\[
\begin{tikzcd}
\underbrace{sK}_{=K}\ar[r, "sk"] & \underbrace{X}_{= s\iota X} \ar[r, "si"] & sI \ar[r, "sj"] & \underbrace{Y}_{= s\iota Y} \ar[r, "sc"] & sK'
\end{tikzcd}
\]
is the canonical decomposition of $\varphi$ in $\SAb$. In particular, $\SAb$ is an abelian category. 

\proof

We'll just verify that $sK'$ is indeed the cokernel:

Let $Z\in \Ob\SAb$. Then there exists an exact sequence
\[
\exactshort{\Hom_{\PAb}(K',\iota Z)}{}{\Hom_{\PAb}(Y,\iota Z)}{}{\Hom_{\PAb}(X,\iota Z)}
\]
(this is equivalent to saying $K'$ is $\coker\varphi$ in $\PAb$). By adjunction,
\[
\exactshort{\Hom_{\SAb}(sK', Z)}{}{\Hom_{SAb}(\underbrace{sY}_{=Y}, Z)}{}{\Hom_{\SAb}(sX,Z)}
\]
is exact. Other parts are exercises

\qed

\defn

Let $A$ be an abelian group, $Y$ a topological space. 

\begin{enumerate}[label=(\alph*)]

\item The \underline{constant presheaf $\A$ on $Y$} is $\A(U) = A$ for all open $U \subseteq Y$, and $\gamma_{U,V} = \Id_A$ for any open $V \subseteq U$. 

\item The \underline{constant sheaf $\mc{A}$ on Y} is $s\A$.

(Check: for connected $U$, $\mc{A}(U) = A$)

\item A sheaf $\mc{F}$ is \underline{locally constant} if any point has a neighborhood $U$ such that $\mc{F}|_U$ is a constant sheaf. (for open $V \subset U \subset Y$, $\mc{F}|_U(V) \eqdef \mc{F}(V)$) (keyword: representation of $\pi_1$ and local systems)

\end{enumerate}

\subsection*{Germs and stalks}

\defn

The \underline{stalk of a (pre)sheaf at a point $y$} is
\[
\mc{F}_y \eqdef \lim_{\longrightarrow\atop{V\ni y}}\mc{F}(V)
\]
with the limit taken with respect to inclusion.

Concretely, $\mc{F}_y \eqdef \frac{\{(s, V) \mid y \in V, s \in \mc{F}(V)\}}{\sim}$, where $(s, V) \sim (s', V')$ if there exists a $W \subseteq V \cap V'$ such that $\gamma_{V,W}(s) = \gamma_{V',W}(s')$. 

Such an equivalence class is called a \underline{germ}

Remark:[GM, I.5.5, I.5.6]

If we have an exact sequence of sheaves
\[
\exactshort{\mc{F}'}{}{\mc{F}}{}{\mc{F}''}
\]
if and only if for any $y \in Y$, 
\[
\exactshort{\mc{F}_y'}{}{\mc{F}_y}{}{\mc{F}_y''}
\]
is exact. This can be verified as an extremely important exercise.


\defn

For $s \in \mc{F}(U)$, the \underline{support of $s$}, $\supp s$, is the closure of the set of points at which the germ of $s$ is not zero. 

Remark:

In the definition of a stalk, we can replace a point $y$ by a closed subset $Z$ of $Y$.

\section*{Lecture 4, 4/10/23}

\subsection*{Functors in abelian categories}

\defn\,

\begin{enumerate}[label=(\alph*)]

\item Let $\ms{C}, \ms{C}'$ be additive categories. A functor $F:\ms{C}\to\ms{C}'$ is \underline{an additive functor} if all maps $F:\Hom_{\ms{C}}(X, Y) \to \Hom_{\ms{C}'}(FX,FY)$ is a homomorphism of abelian groups.

\item A \underline{complex} in $\ms{C}$ is a sequence
\[X^\cdot :
\begin{tikzcd}
\cdot \ar[r, "d^{n - 1}"] & X^n \ar[r, "d^n"] & X^{n + 1}\ar[r, "d^{n + 1}"] & \cdots \\
\end{tikzcd}
\]
with $d^n\circ d^{n - 1} = 0$ for all $n$

\item Now assume $\ms{C},\ms{C}'$ are abelian categories. If we have a complex, then, because $d^n\circ d^{n + 1} =0 $, the universal properties of the kernel and cokernel (as well as their existence, which is guaranteed because we are in an abelian category), guarantee us unique maps $a^n, b^{n + 1}$ making the diagram commute:
\[
\begin{tikzcd}
 & \coker d^n \ar[dr, dotted, "b^{n + 1}"] & \\
X^n \ar[r, "d^n"] \ar[dr, "a^n"', dotted] & X^{n + 1}\ar[u] \ar[r, "d^{n + 1}"] & X^{n + 2} \\
& \ker d^{n + 1} \ar[u] & \\
\end{tikzcd}
\]
 The \underline{$(n + 1)$-cohomology of $X^\cdot$} is
\[
H^{n + 1}(X^\cdot) \eqdef \coker a^n = \ker b^{n + 1}
\]
this equality can be verified as an exercise.. 

\item $X^\cdot$ is \underline{acyclic at $X^b$} if $H^n(X^\cdot) = 0$.

\item $X^\cdot$ is \underline{exact/acyclic} if it is acyclic at $X^n$ for all $n$

\item An additive functor $F:\ms{C}\to\ms{C}'$ is \underline{exact} if it sends a short exact sequence 
\[
\exactshort{X}{}{Y}{}{Z}
\]
to a short exact sequence
\[
\exactshort{FX}{}{FY}{}{FZ}
\]
It is \underline{left exact} if
\[
\begin{tikzcd}
0 \ar[r] & FX \ar[r] & FY \ar[r] & FZ
\end{tikzcd}
\]
is exact, and \underline{right exact} if
\[
\begin{tikzcd}
FX \ar[r] & FY \ar[r] & FZ \ar[r] & 0 
\end{tikzcd}
\]
is exact.

\end{enumerate}

\exm

Let $\ms{C}$ be an abelian category, and consider $\Hom_{\ms{C}}(Y,-):\ms{C}\to\Ab$ and $\Hom_{\ms{C}}(-,Y): \ms{C}^\circ\to\Ab$, where $\ms{C}^\circ$ means the opposite category of $\ms{C}$ (this is just saying this functor is contravariant). These two morphisms are both left-exact.

\exm

For a fixed ring $R$, consider $\RMod$, the category of left $R$-modules, and $Y$ a right $R$-module (so an object of Mod-$R$). Then we have a functor
\[
Y\otimes_{R}:\RMod\to\Ab
\]
which is right-exact.

\prop

Let $X$ be a topological space, and fix an open set $U \subseteq X$. Consider $\SAb$, the category of abelian sheaves on $X$. The functor $\SAb\to \Ab$ given by $\mc{F}\mapsto\mc{F}(U)$ is an additive functor which is left exact.

\proof

Let $\iota:\SAb\to\PAb$ be the inclusion of sheaves into presheaves. This is left exact, which follows from the fact that the kernel of a morphism of sheaves is again a sheaf. The kernel doesn't need sheafification! Now $\PAb\to\Ab: \mc{F}\mapsto\mc{F}(U)$ is exact by definition. The composition of a left exact and an exact functor is left exact, so we are done. 

\qed

From now on, we will always be working in an abelian category unless otherwise stated.

\defn\,

\begin{enumerate}[label=(\alph*)]

\item An object $Y$ is \underline{projective} if $\Hom_{\ms{C}}(Y,-)$ is exact.

\item An object $Y$ is \underline{injective} if $\Hom_{\ms{C}}(-,Y)$ is exact

\item A right module-$R$ $Y$ is \underline{flat} if $Y\otimes_R-$ is exact.

\end{enumerate}

\subsection*{Direct images}

\defn

Let $f:M\to N$ be a continuous map of topological spaces, and let $\mc{F}$ be a sheaf on $M$. The \underline{direct image $f_\star\mc{F}$} (written as $f_\cdot \mc{F}$ in Gelfond-Manin) is defined as follows. 

For any open $U \subseteq N$, $f^{-1}(U)$ is open in $M$. So we simply define
\[
f_\star\mc{F}(U) \eqdef \mc{F}(f^{-1}(U))
\]
and restriction for $V \subseteq U$ induced from $\gamma_{f^{-1}(U),f^{-1}(V)}$

Exercise: verify that this is indeed a sheaf!

\prop\,

\begin{enumerate}[label=(\alph*)]

\item Let $f:M\to\{1\}$ be the constant map. Then $f_\star\mc{F} = \Gamma(M,\mc{F}) = \mc{F}(M)$.

\item Let $i:M\to N$ be the inclusion of a closed subspace $M$ of $N$. Then
\[
(i_\star\mc{F})_x = \begin{cases} \mc{F}_x & x \in M \\ 0 & x\not\in M\end{cases}
\]

Some call this an ``extension by zero." (If $i:M\into N$ is the inclusion of an open subset $M$ of $N$, then $i_*\mc{F}$ may have nonzero stalk at $x \in N\setminus M$. Prove this as an exercise!)

\item $f_*:\SAb(M)\to\SAb(N)$ is a functor, $(fg)_* = f_* g_*$

\proof

\qed

\end{enumerate}

\section*{Lecture 5, 4/12/23}

Inverse image. Let $f:M\to N$ be continuous. 

\defn For $\mc{F} \in \SAb_N$, first define $f_p^\star \mc{F}$ as a presheaf:
\[
U\mapsto \mc{F}(f(U)) \eqdef \lim_{\to\atop{N \supset V \supset f(U)}}\mc{F}(V)
\]
where $V$ is open.
Then take $f^\star\mc{F} \eqdef s(f_p^\star \mc{F})$

Exercise: for all $x \in M, (f^\star\mc{F})_{x} = \mc{F}_{f(x)}$

\prop

We have an adjunction $f^\star \dashv f_\star$, 
\[
\Hom_{\SAb_M}(f^\star\mc{F}, \mc{G}) \cong \Hom_{\SAb_N}(\mc{F}, f_*\mc{G})
\]

\proof

We know $s \dashv \iota$ by construction. So, we need only show
\[
\Hom_{\PAb_M}(f_p^\star\mc{F}, \mc{G}) \cong \Hom_{\PAb_N}(\mc{F}, f_\star\mc{G})
\]

We establish a functorial morphism: $\mc{F}\to f_\star f_p^\star \mc{F}$. 

Let $\mc{G} = f_p^\star \mc{F}$, 
\[
\Hom(f_p^\star\mc{F}, f_p^\star\mc{F}) \cong \Hom(\mc{F}, f_\star f_p^\star\mc{F})
\]

Note: for $V \subseteq N$ open, $\mc{F}(V) \to f_p^\star (f^{-1}(V))$

But $V$ is open in $N$ containing $f(f^{-1}(V))$. So we have restriction

?????????????

Exercise: Check morphism of presheaves.

This is compatible with restrictions and gives us a presheaf morphism $i_{\mc{F}}:\mc{F}\to f_\star f_p^\star \mc{F}$. 

This induces the isomorphism in the statement. 
\[
(\psi:f_p^\star\mc{F}\to \mc{G}) \longrightarrow ((f_\star\psi)\circ i_{\mc{F}}: \mc{F}\to f_*\mc{G}
\]
The other direction uses $f_p^\star f_\star \mc{G} \to \mc{G}$

Exercise: construct this and check inverse

\qed

\prop 

Let $\ms{C},\ms{D}$ be Abelian categories. Let $F:\ms{C}\to\ms{D}$ and $G:\ms{D}\to\ms{C}$ be adjoint functors, $F\dashv G$. 

\thm

$F$ is right exact, and $G$ is left exact

\proof

We will just check $G$ is left exact. 

Let $\exactshort{Y'}{f}{Y}{g}{Y''}$ be a short exact sequence.

Apply the left exact functor $\Hom_{\ms{D}}(Fx, -)$ for all $x\in\ms{C}$. We have the diagram
\[
\begin{tikzcd}
0 \ar[r] & \Hom_{\ms{D}} \ar[d, equal] \ar[r] & \Hom_{\ms{D}}(FX, Y') \ar[d, equal] \ar[r] & \Hom_{\ms{D}}(FX, Y'')\ar[d, equal] \\
0 \ar[r] & \Hom_{\ms{C}}(X, GY') \ar[r] & \Hom_{\ms{C}}(X, GY) \ar[r] & \Hom_{\ms{C}}(X, GY'')
\end{tikzcd}
\]
This is exact for all $X\in\ms{C}$ iff $\begin{tikzcd} 0 \ar[r] & GY' \ar[r] & GY \ar[r] & GY''\end{tikzcd}$ is exact

\prop

In $\SAb, f_\star$ is left exact, $f^\star$ is exact. 

\proof 

By exercise, $f^\star$ is exact on stalks.

\subsection*{Direct images with compact support}

GM, $III8.7->8.10$

All topological spaces are assumed to be locally compact and first countable, meaning every point has a countable neighborhood basis. 

Recall: A morphism of topological spaces is proper if the preimage of compact sets are compact. 

\defn

Let $f:X\to Y$, $\mc{F}$ a sheaf on $X$. Let $U \subseteq Y$ be open. We define
\[
F_!\mc{F}(U) = \{s \in \Gamma(f^{-1}(U),\mc{F}) \mid f:\operatorname{supp}(s) \to U\text{ is proper }\}
\]

\defn

Let $s \in \Gamma(V,\mc{G})$. Then $\supp(s) = \overline{\{x\in V \mid \bar{s} \neq 0 \in \mc{G}_x\}}$, $\Gamma(V, \mc{G}) \to \mc{G}$, $s\mapsto \bar{s}$

\lem\,

\begin{enumerate}[label=(\alph*)]

\item $f_!\mc{F}$ is a subsheaf of $f_*\mc{F}$

\item $f_!$ is a left exact functor.

\end{enumerate}

\proof

\qed



\section*{Lecture 6, 4/14/23}

Let $f:X\to Y$ be continuous, and $\mc{F}$ be a sheaf on $X$. Recall the definition of $f_!\mc{F}$ :
\[
f_!\mc{F}(U) = \{s \in \underbrace{\Gamma(f^{-1}(U),\mc{F})}_{=f_\star\mc{F}(U)} \mid f:\operatorname{supp}(s) \to U\text{ is proper }\}
\]
where $\supp(s)$ is the closure of the set of points where $\bar{s}$, the germ of $s$, $\bar{s}$, is not zero. 

\thm\,

\begin{enumerate}[label=(\alph*)]

\item $f_!\mc{F}$ is a \underline{subsheaf} of $f_*\mc{F}$. 

\item $f_!$ is a left exact functor "direct image with compact support"

\end{enumerate}

\proof\,

\begin{enumerate}[label=(\alph*)]

\item $f_!\mc{F}$ is clearly a subpresheaf of $f_\star\mc{F}$. Any set of compatible sections of $f_!\mc{F}$ glue uniquely to a section of $f_\star\mc{F}$. This comes down to a topological statement. 

Exercise: For $(U_i)$ open subsets of $Y$, $f_i:V_i\to U_i$ is proper, then $f: \cup V_i \to \cup U_i$ is proper. 
\[
f^{-1}(K) = \cup f_i^{-1}(V_i \cap K)
\]

\item 

\end{enumerate}

\qed

\subsection*{Sections with compact support}

Consider the special case $f:X\to\{1\}$, the one point space. Then $f_!\mc{F}$ is the set of sections $s \in \mc{F}(X)$ such that $\supp(s)$ is compact. 

\,

Denote this by $\Gamma_c(X, \mc{F})$
\[
\begin{tikzcd}
X \ar[r, "f"] & Y \\
f^{-1}(y) \ar[u, "i", hook] \ar[r] & y \ar[u, hook]
\end{tikzcd}
\]

\prop

The stalk of $f_!\mc{F}$ at $y \in Y$ is isomorphic to 
\[
\Gamma_C(f^{-1}(y), \underbrace{\mc{F}|_{f^{-1}(y)}}_{=i^\star\mc{F}})
\]

\proof

First construct $\varphi:(f_!\mc{F})_y \to \Gamma_c(f^{-1}(y), \mc{F}|_{f^{-1}(y)})$.

Let $s \in f_!\mc{F}_y$ and choose a representative $\tilde{s} \in \Gamma(f^{-1}(U), \mc{F})$ with $U$ an open neighborhood of $y$, and $\supp\tilde{s}\to U$ proper. 

Then: $\tilde{s}|_{f^{-1}(y)}$ is in $\Gamma_c(f^{-1}(y), \underbrace{\mc{F}|_{f^{-1}(y)}}_{=i^\star\mc{F}})$

Exercise: $\varphi(s) \eqdef \tilde{s}|_{f^{-1}(y)}$ only depends on $s$. 

We now show $\varphi$ is injective. Suppose $\varphi(s) = 0$. Then $\supp(\tilde{s}) \cap f^{-1}(y) = \varnothing$. So $y\not\in f(\supp\tilde{s})$. But $f(\supp\tilde{s})$ is closed (proper + locally compact)

So $s = 0$. 

To show $\varphi$ is surjective: choose a local basis $V_i \ni y$ with $\cap V_i = y$. Then $f^{-1}(y) = \cap f^{-1}(U_i)$

Exercise:

Locally compadct implies $\Gamma_c(f^{-1}(y), \mc{F}|_{f^{-1}(y)} = \lim_{\to} A_i$ where
\[
A_i = \{t \in \Gamma(f^{-1}(U_i), \mc{F}) \mid \supp t = K \cap f^{-1}(U_i) \text{ for some compact }K \subseteq X\}
\]
\qed

\exm

\begin{enumerate}

\item Let $i:U\into X$ be open, $\mc{F}$ a sheaf on $U$. 
\[
(i_!\mc{F})_x = \begin{cases} \mc{F}_x, & x \in U \\ 0, & x \in U \end{cases}
\]
``extend by 0"

\item $j:V\to X$ proper (in particular, closed embedding), $j_!\mc{G} = j_*\mc{G}$.


\end{enumerate}


\subsection*{Derived Categories via Localizations}

\defn

Let $f:K^\cdot\to L^\cdot$ be a morphism of complexes in an Abelian category $\mc{A}$. $f$ is a \underline{quasi-isomorphism} if the induced morphism $H^n(f):H^n(K^\cdot)\to H^n(L^\cdot)$ is an isomorphism for all $n$. 

\defn

Let $\mc{A}$ be an Abelian category, $\Kom(\mc{A})$ the category of complexes in $\mc{A}$. The \underline{derived category of $\mc{A}$} is a category $D(A)$ and a functor $Q:\Kom(A)\to D(A)$ such that
\begin{enumerate}[label=(\alph*)]

\item $Q(f)$ is an isomorphism for any quasi-isomorphism $f$. 

\item $Q$ is universal in the following sense. Suppose that $g:\mc{A}\to \mc{D}$ is a functor such that $g(f)$ is an isomorphism for any quasi-isomorphism $f$. Then there is a unique functor $\bar{f}:D(\mc{A})\to\mc{D}$, making the diagram commute:
\[
\begin{tikzcd}
\Kom(\mc{A}) \ar[rr, "f"] \ar[dr, "Q"'] && \mc{D} \\
& D(\mc{A}) \ar[ur, "\bar{f}"'] & \\
\end{tikzcd}
\]

\end{enumerate}

\thm

Every abelian category admits a derived category.

\proof

\qed

\section*{Lecture 7, 4/17/23}

Homework hint:

Let $\mc{A}$ be an abelian category with $P$ a projective generator. We wish to define an equivalence with $R$-mod, where $R = \Hom_{\mc{A}}(P,P)$. Try to construct an equivalence going the other way as follows:

We want a projective resolution of $M$ 
\[
\begin{tikzcd}
R^{\oplus J} \ar[r] & R^{\oplus I} \ar[r] & M \ar[r] & 0 \\
\end{tikzcd}
\]
We want to use this to get a morphism $P^{\oplus J} \to P^{\oplus I}$. You can do this by taking the identity and looking at the image (?).

\,

Now we continue. 

Remark: We can define $\Kom^+(\mc{A})$ to be all the chain complexes in $\mc{A}$ that are bounded on the right. That is, $\Kom^+(\mc{A}) = \{K^\cdot \mid K^i = 0$ for $i << 0\}$. $\Kom^{-}(\mc{A})$ is defined similarly. 
\,
$\Kom^b(\mc{A}) = \Kom^+(\mc{A}) \cap \Kom^-(\mc{A})$. 

These are each abelian categories, so we can consider $D^+(\mc{A}), D^-(\mc{A}), D^b(\mc{A})$, but this is for later. 

\subsection*{\underline{Construction of $D(\mc{A})$}}

This is similar to ``localization of rings"

\defn

A class of morphism $S \subset \Mor \ms{B}$ is said to be \underline{localizing} if
\begin{enumerate}[label=(\alph*)]

\item $S$ is closed under composition: $\Id_X\in S$ for all $X \in \Ob\ms{B}$ and $s \circ t \in S$ for all $s, t \in S$ such that the composition is defined. 

\item Extension: for all $f \in \Mor \ms{B}$, $s \in S$, there exists $g \in \Mor \ms{B}, t \in S$ such that we can make one of the following diagrams commute:
\[
\begin{tikzcd}
W \ar[r, dotted, "g"] \ar[d, dotted, "t"'] & Z \ar[d, "s"] \\
X \ar[r, "f"'] & Y
\end{tikzcd}
\]
or
\[
\begin{tikzcd}
W & \ar[l, dotted, "g"'] Z \\
X \ar[u, dotted, "t"]  & \ar[l, "f"] \ar[u, "s"'] Y
\end{tikzcd}
\]

\item If $f, g \in \Mor(X, Y)$, the existence of $s \in S$ with $sf = sg$ is equivalent to the existence of $t \in S$ such that $ft = gt$

\end{enumerate}

\defn

Given a category $\ms{B}$ and class of morphisms $S$ which is localizing, we introduce a category $\ms{B}[S^{-1}]$, with $\Ob \ms{B}[S^{-1}] = \Ob\ms{B}$ and 
\begin{enumerate}[label=(\alph*)]

\item A morphism $X\to Y$ in $\ms{B}[S^{-1}]$ is the equivalence class of \underline{roofs} of the form 
\[
\begin{tikzcd}
& X' \ar[dl, "s"'] \ar[dr, "f"] & \\
X & & Y\\
\end{tikzcd}
\]
with $s \in S, f \in \Mor\ms{B}$. Two roofs are equivalent if there exists a third roof making the following diagram commute:
\[
\begin{tikzcd}
& & X\ar[dl, dotted, "y"'] \ar[dr, dotted, "h"] && \\
& X'\ar[dl, "s"']\ar[drrr,"f"'] && X''\ar[dlll, "t"] \ar[dr, "g"]& \\
X & && &Y \\
\end{tikzcd}
\]
With $sr = th, fr = gh, y \in S$. $\Id_X$ is the class of
\[
\begin{tikzcd}
& X \ar[dr, "\Id_X"] \ar[dl, "\Id_X"'] & \\
X & & X\\
\end{tikzcd}
\]

\item The composition of roofs $(s, f)$ and $(t, g)$ is the class of $(st', gf')$
\[
\begin{tikzcd}
& & X''\ar[dl, "t'"'] \ar[dr, "f'"] \ar[ddll, "st'"', bend right = 40]\ar[ddrr, "gf'", bend left = 40]& & \\
& X' \ar[dl, "s"'] \ar[dr, "f"]& & Y'\ar[dl, "t"'] \ar[dr, "g"] & \\
X& & Y& & Z
\end{tikzcd}
\]
$gt^{-1}fs^{-1} = gf't'^{-1}s^{-1} = (gf)(st')^{-1}$

\end{enumerate}

\thm

This is in fact a well defined equivalence relation

\proof

Symmetry and reflexivity are easy, so we will just show transitivity. Suppose $(s, f) \sim (t, g), (t, g) \sim (u, e)$. We want to show $(s, f) \sim (u, e)$. 

So we have
\[
\begin{tikzcd}
& & Z'\ar[dl, "r"']\ar[dr, "h"] & & Z \ar[dl, "p"'] \ar[dr, "i"]& & \\
& X' \ar[dl, "s"']\ar[drrrrr,"f"']& & \ar[dlll, "t"']X'\ar[drrr, "g"] & & X'' \ar[dr, "e"]\ar[dlllll,"u"]& \\
X & & & & & & Y
\end{tikzcd}
\]
$s, t, u, r, p \in S$
First consider
\[
\begin{tikzcd}
Z' \ar[d, "sr"'] & W \ar[l, dotted, "v"'] \ar[d, dotted, "k"] \\
X & Z'' \ar[l, "tp"]
\end{tikzcd}
\]
`````````with $sr \in S$. Then $thv = srv = tpk$, so by $c$ there exists $w$ such that $hvw = pkw$ for some $w:Z'''\to W$, $w \in S$. We now build the roof
\[
\begin{tikzcd}
& & Z''' \ar[dl, "rvw"']\ar[dr, "ikw"]& & \\
& X' \ar[dl, "s"'] \ar[drrr, "f"']& & X''\ar[dr, "e"] \ar[dlll, "u"] & \\
X & & & & Y 
\end{tikzcd}
\]

\qed

\defn

The category $\ms{B}[S^{-1}]$ is called a \underline{localization of $\ms{B}$}. 

Let $F:\ms{B}\to\ms{B}[S^{-1}]$ map each object to its equivalence class. We define $F(f)$ to be the equivalence class of the roof
\[
\begin{tikzcd}
& X \ar[dl, "\Id_X"'] \ar[dr, "f"] & \\
X & & Y \\
\end{tikzcd}
\]
This satisfies the universal property:

Let $T:\ms{B}\to\ms{D}$ be such that $T(s)$ is an isomorphism for all $s \in S$. Then there exists a unique factorization
\[
\begin{tikzcd}
\ms{B}\ar[dr, "F"'] \ar[rr, "T"] & & \ms{D} \\
& \ms{B}[S^{-1}] \ar[ur, dotted, "\exists!"'] &
\end{tikzcd}
\]






























\end{document}
